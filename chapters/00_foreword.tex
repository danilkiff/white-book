\chapter*{Foreword}

This work was born from a growing unease -- a recognition that something
subtle but profound was shifting in the way humans know, reason, and
understand. It is not a lament for the past, nor a rejection of
technological progress. It is an attempt to think carefully and
systematically about a transformation that has unfolded largely without
deliberate reflection.

The goal of this book is not to offer simple answers. It is to map the
landscape of cognitive change, to understand its mechanisms, and to
suggest paths forward that respect the complexity of both human minds
and human cultures.

The intended audience is broad: anyone who cares about the future of
knowledge, reasoning, and autonomy -- educators, technologists,
policymakers, scholars, and reflective individuals across all fields.

This book demands a certain patience from its readers. It asks that they
slow down, think deeply, and engage with complexity without rushing to
closure. Such patience is not an incidental courtesy; it is a small
enactment of the very cognitive virtues that the book seeks to defend
and cultivate.

If this work succeeds, it will not be by providing final answers. It
will be by helping to reopen a conversation -- about what it means to
know, to reason, and to build futures worthy of human dignity.